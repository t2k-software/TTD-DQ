\documentclass{beamer}
\usepackage[english] {babel}
\usepackage[T1]      {fontenc}
\usepackage{amsmath, amsfonts, graphicx}
\usepackage{bibunits, tikz, version}
\usetheme[pageofpages=of,% String used between the current page and the
% total page count.
alternativetitlepage=true,% Use the fancy title page.
titlepagelogo=images/t2k_logo_medium,% Logo for the first page.
]{Torino}
\usecolortheme{nouvelle}

\title{ECal Data Quality Assessment}
\subtitle{Data Quality Checks for the period : 22/05/2016 - 28/05/2016}
\author{Pierre Lasorak}
\date{\today}

\logo{\includegraphics[height=0.1\paperheight]{images/t2k_logo_medium.png}}

\institute{\includegraphics[scale=0.032]{images/QM_logo.png}}

\newcommand{\newCommandName}{0522-0528}

\begin{document}
\maketitle

\begin{frame}{Overview}
  \tableofcontents
\end{frame}

\section{Beam Timing}
\begin{frame}{Beam Timing 1/6: RMM00 \& RMM01}
  \begin{center}
    \includegraphics[width=0.9\textwidth]{../\newCommandName /BeamTiming/bunchtiming_weekly1.png}
    \begin{footnotesize}
      \begin{itemize}
      \item The shift is due to a change of neutrino timing.
      \end{itemize}
    \end{footnotesize}
  \end{center}
\end{frame}
\begin{frame}{Beam Timing 2/6: RMM02 \& RMM03}
  \begin{center}
    \includegraphics[width=0.9\textwidth]{../\newCommandName /BeamTiming/bunchtiming_weekly2.png}
  \end{center}
\end{frame}
\begin{frame}{Beam Timing 3/6: RMM04 \& RMM05}
  \begin{center}
    \includegraphics[width=0.9\textwidth]{../\newCommandName /BeamTiming/bunchtiming_weekly3.png}
  \end{center}
\end{frame}
\begin{frame}{Beam Timing 4/6: RMM06 \& RMM07}
  \begin{center}
    \includegraphics[width=0.9\textwidth]{../\newCommandName /BeamTiming/bunchtiming_weekly4.png}
  \end{center}
\end{frame}
\begin{frame}{Beam Timing 5/6: RMM08 \& RMM09}
  \begin{center}
    \includegraphics[width=0.9\textwidth]{../\newCommandName /BeamTiming/bunchtiming_weekly5.png}
  \end{center}	
\end{frame}
\begin{frame}{Beam Timing 6/6: RMM10 \& RMM11}
  \begin{center}
    \includegraphics[width=0.9\textwidth]{../\newCommandName /BeamTiming/bunchtiming_weekly6.png}
  \end{center}
  
\end{frame}

\section{Gain Variation}
\begin{frame}{Gain drift 1/2}
  \begin{center}
    \includegraphics[width=0.45\textwidth]{../\newCommandName /Gain/gaindriftnew0.png}
    \hspace{0.5cm}
    \includegraphics[width=0.45\textwidth]{../\newCommandName /Gain/gaindriftnew1.png}
  \end{center}
\end{frame}

\begin{frame}{Gain drift 2/2}
  \begin{center}
    \includegraphics[width=0.45\textwidth]{../\newCommandName /Gain/gaindriftnew2.png}
    \hspace{0.5cm}
    \includegraphics[width=0.45\textwidth]{../\newCommandName /Gain/gaindriftnew3.png}
  \end{center}
\end{frame}

\section{Pedestal Variation}
\begin{frame}{Pedestal drift 1/4}
  \begin{center}
    \includegraphics[width=0.9\textwidth]{../\newCommandName /Ped/PedDriftRMM0.png}
  \end{center}
\end{frame}
\begin{frame}{Pedestal drift 2/4}
  \begin{center}
    \includegraphics[width=0.9\textwidth]{../\newCommandName /Ped/PedDriftRMM1.png}
  \end{center}
\end{frame}
\begin{frame}{Pedestal drift 3/4}
  \begin{center}
    \includegraphics[width=0.9\textwidth]{../\newCommandName /Ped/PedDriftRMM2.png}
  \end{center}
\end{frame}
\begin{frame}{Pedestal drift 4/4}
  \begin{center}
    \includegraphics[width=0.9\textwidth]{../\newCommandName /Ped/PedDriftRMM3.png}
  \end{center}
\end{frame}

\section{Channels}
\begin{frame}{Total / Dead / Bad / Overflow / Underflow channels (from Gain files)}
  \begin{center}
    \includegraphics[width=0.3\textwidth]{../\newCommandName /Channels/totChannels.png}
    \hspace{0.2cm}
    \includegraphics[width=0.3\textwidth]{../\newCommandName /Channels/DeadChannels.png}
    \hspace{0.2cm}
    \includegraphics[width=0.3\textwidth]{../\newCommandName /Channels/BadChannels.png} \\
    \vspace{0.2cm}
    \includegraphics[width=0.3\textwidth]{../\newCommandName /Channels/OverChannels.png}
    \hspace{0.5cm}
    \includegraphics[width=0.3\textwidth]{../\newCommandName /Channels/underChannels.png}
  \end{center}
\end{frame}
\section{Flags} 
\begin{frame}{Flags}
  \begin{center}
    \begin{itemize}
    \item ECal wasn't in local during this week
    % \end{itemize}
    % \vspace{0.2cm}
    % \begin{scriptsize}
    %   \begin{tabular}{c|c|c|c|l}
    %     \hline \hline
    %     First run & Starting time (GMT)  & Last run & Ending time (GMT) & Cause \\
    %     \hline
    %     12538|0 & 12/05 @ 02:08:02 & 12538|4 & 12/05 @ 04:04:12 &  Calibration \\
    %     \hline \hline
    %   \end{tabular}
    % \end{scriptsize}
    % \begin{itemize}
    % \item A bad flag will be set for all the detector during this period (flag=4095).
    \item No RMM skew detected.
    \end{itemize}
  \end{center}
\end{frame}

\begin{frame}{Flags - Plots}
  \begin{center}
          \begin{minipage}[t]{0.48\linewidth}
            \begin{center}
      Beam\\$ $ \\
      \includegraphics[scale=0.23, angle = 0]{../\newCommandName /validation_ECAL_beam.png}
    \end{center}	
  \end{minipage}
  \begin{minipage}[t]{0.48\linewidth}
    \begin{center}		
      Cosmics\\$ $\\
      \includegraphics[scale=0.23, angle = 0]{../\newCommandName /validation_ECAL_cosmic.png}
    \end{center}
  \end{minipage}
\end{center}
\end{frame}

\end{document}

\begin{frame}{Note}
  \begin{center}
    \begin{footnotesize}
      \begin{itemize}
      \item When the number of Dead Channels oscillate with 1 or 2 it is usually not so important:
        
        These channels usually go in the Bad Channels (you can see the anti-correlation).
      \item Digging the documentation / code I could fine:
        \begin{itemize}
          \begin{footnotesize}
          \item DEAD channel is when only pedestal peak is visible (HV trim is set to 255)
          \item BAD channel is when channel saturates {\em or only pedestal peak is visible}
          \item In practice, this is implemented as:
            \begin{itemize}
              \begin{footnotesize}
                \item Dead channel: gain <= 0 (no p.e. peak found)
                \item Bad channel: gain > 0 \& error < 0 (something went wrong with the search of the 2 p.e. peaks: too far or too close $\rightarrow$ saturation)
              \end{footnotesize}
            \end{itemize}
          \end{footnotesize}
        \end{itemize}
      \item These 2 definitions overlap.
      \item Overflow and Underflow Channels correspond to the channels where Gain are off the plots (slides 10 and 11).
      \end{itemize}
    \end{footnotesize}
  \end{center}
\end{frame}
